A cross-\/platform library that provides a simple A\+P\+I to read and write I\+N\+I-\/style configuration files. It supports data files in A\+S\+C\+I\+I, M\+B\+C\+S and Unicode. It is designed explicitly to be portable to any platform and has been tested on Windows, Win\+C\+E and Linux. Released as open-\/source and free using the M\+I\+T licence.

\section*{Feature Summary}


\begin{DoxyItemize}
\item M\+I\+T Licence allows free use in all software (including G\+P\+L and commercial)
\item multi-\/platform (Windows 95/98/\+M\+E/\+N\+T/2\+K/\+X\+P/2003, Windows C\+E, Linux, Unix)
\item loading and saving of I\+N\+I-\/style configuration files
\item configuration files can have any newline format on all platforms
\item liberal acceptance of file format
\begin{DoxyItemize}
\item key/values with no section
\item removal of whitespace around sections, keys and values
\end{DoxyItemize}
\item support for multi-\/line values (values with embedded newline characters)
\item optional support for multiple keys with the same name
\item optional case-\/insensitive sections and keys (for A\+S\+C\+I\+I characters only)
\item saves files with sections and keys in the same order as they were loaded
\item preserves comments on the file, section and keys where possible.
\item supports both char or wchar\+\_\+t programming interfaces
\item supports both M\+B\+C\+S (system locale) and U\+T\+F-\/8 file encodings
\item system locale does not need to be U\+T\+F-\/8 on Linux/\+Unix to load U\+T\+F-\/8 file
\item support for non-\/\+A\+S\+C\+I\+I characters in section, keys, values and comments
\item support for non-\/standard character types or file encodings via user-\/written converter classes
\item support for adding/modifying values programmatically
\item compiles cleanly in the following compilers\+:
\begin{DoxyItemize}
\item Windows/\+V\+C6 (warning level 3)
\item Windows/\+V\+C.\+N\+E\+T 2003 (warning level 4)
\item Windows/\+V\+C 2005 (warning level 4)
\item Linux/gcc (-\/\+Wall)
\item Windows/\+Min\+G\+W G\+C\+C
\end{DoxyItemize}
\end{DoxyItemize}

\section*{Documentation}

Full documentation of the interface is available in doxygen format.

\section*{Examples}

These snippets are included with the distribution in the file snippets.\+cpp.

\subsubsection*{S\+I\+M\+P\+L\+E U\+S\+A\+G\+E}


\begin{DoxyCode}
1 \{c++\}
2 CSimpleIniA ini;
3 ini.SetUnicode();
4 ini.LoadFile("myfile.ini");
5 const char * pVal = ini.GetValue("section", "key", "default");
6 ini.SetValue("section", "key", "newvalue");
\end{DoxyCode}


\subsubsection*{L\+O\+A\+D\+I\+N\+G D\+A\+T\+A}


\begin{DoxyCode}
1 \{c++\}
2 // load from a data file
3 CSimpleIniA ini(a\_bIsUtf8, a\_bUseMultiKey, a\_bUseMultiLine);
4 SI\_Error rc = ini.LoadFile(a\_pszFile);
5 if (rc < 0) return false;
6 
7 // load from a string
8 std::string strData;
9 rc = ini.LoadData(strData.c\_str(), strData.size());
10 if (rc < 0) return false;
\end{DoxyCode}


\subsubsection*{G\+E\+T\+T\+I\+N\+G S\+E\+C\+T\+I\+O\+N\+S A\+N\+D K\+E\+Y\+S}


\begin{DoxyCode}
1 \{c++\}
2 // get all sections
3 CSimpleIniA::TNamesDepend sections;
4 ini.GetAllSections(sections);
5 
6 // get all keys in a section
7 CSimpleIniA::TNamesDepend keys;
8 ini.GetAllKeys("section-name", keys);
\end{DoxyCode}


\subsubsection*{G\+E\+T\+T\+I\+N\+G V\+A\+L\+U\+E\+S}


\begin{DoxyCode}
1 \{c++\}
2 // get the value of a key
3 const char * pszValue = ini.GetValue("section-name", 
4     "key-name", NULL /*default*/);
5 
6 // get the value of a key which may have multiple 
7 // values. If bHasMultipleValues is true, then just 
8 // one value has been returned
9 bool bHasMultipleValues;
10 pszValue = ini.GetValue("section-name", "key-name", 
11     NULL /*default*/, &amp;bHasMultipleValues);
12 
13 // get all values of a key with multiple values
14 CSimpleIniA::TNamesDepend values;
15 ini.GetAllValues("section-name", "key-name", values);
16 
17 // sort the values into the original load order
18 values.sort(CSimpleIniA::Entry::LoadOrder());
19 
20 // output all of the items
21 CSimpleIniA::TNamesDepend::const\_iterator i;
22 for (i = values.begin(); i != values.end(); ++i) \{ 
23     printf("key-name = '%s'\(\backslash\)n", i->pItem);
24 \}
\end{DoxyCode}


\subsubsection*{M\+O\+D\+I\+F\+Y\+I\+N\+G D\+A\+T\+A}


\begin{DoxyCode}
1 \{c++\}
2 // adding a new section
3 rc = ini.SetValue("new-section", NULL, NULL);
4 if (rc < 0) return false;
5 printf("section: %s\(\backslash\)n", rc == SI\_INSERTED ? 
6     "inserted" : "updated");
7 
8 // adding a new key ("new-section" will be added 
9 // automatically if it doesn't already exist)
10 rc = ini.SetValue("new-section", "new-key", "value");
11 if (rc < 0) return false;
12 printf("key: %s\(\backslash\)n", rc == SI\_INSERTED ? 
13     "inserted" : "updated");
14 
15 // changing the value of a key
16 rc = ini.SetValue("section", "key", "updated-value");
17 if (rc < 0) return false;
18 printf("key: %s\(\backslash\)n", rc == SI\_INSERTED ? 
19     "inserted" : "updated");
\end{DoxyCode}


\subsubsection*{D\+E\+L\+E\+T\+I\+N\+G D\+A\+T\+A}


\begin{DoxyCode}
1 \{c++\}
2 // deleting a key from a section. Optionally the entire
3 // section may be deleted if it is now empty.
4 ini.Delete("section-name", "key-name", 
5     true /*delete the section if empty*/);
6 
7 // deleting an entire section and all keys in it
8 ini.Delete("section-name", NULL);
\end{DoxyCode}


\subsubsection*{S\+A\+V\+I\+N\+G D\+A\+T\+A}


\begin{DoxyCode}
1 \{c++\}
2 // save the data to a string
3 rc = ini.Save(strData);
4 if (rc < 0) return false;
5 
6 // save the data back to the file
7 rc = ini.SaveFile(a\_pszFile);
8 if (rc < 0) return false;
\end{DoxyCode}
 