Fire is a framework for modeling and simulation. It includes various reusable tools and utilities for solving computational physics problems. This includes
\begin{DoxyItemize}
\item parsers
\item steppers
\item solvers
\item a finite element library
\end{DoxyItemize}

It also includes domain-\/specific classes and tools for
\begin{DoxyItemize}
\item kinetic rate theory calculations in nuclear astrophysics
\item heat conduction and diffusion
\item sintering
\item welding
\end{DoxyItemize}

\subsection*{Documentation}

Documentation in Fire is generated via Doxygen by running \char`\"{}make doc\char`\"{} during the build. The documentation is viewable \href{http://www.jayjaybillings.com/fire}{\tt online}. The full A\+PI documentation is available at \href{http://www.jayjaybillings.com/fire/api/html/}{\tt the A\+PI reference page}.

You can run \char`\"{}make doc\char`\"{} from build directory to generate the A\+PI documentation. Doxygen handles most of the required documentation without developer intervention. This means that in some cases there may be classes that seem to have minimal documentation in the source, like classes that implement interfaces and provide no additional functionality, but are in fact quite well documented by Doxygen. Most I\+D\+Es will also auto-\/generate descriptions for developers too, so the author(s) see no need to cover every piece of code with comments.

\subsection*{Prerequisites}

You will need git and cmake to build Fire.

\subsection*{Checkout and build}

From a shell, execute the following commands to compile the code\+:


\begin{DoxyCode}
git clone https://github.com/jayjaybillings/fire
mkdir fire-build
cd fire-build
cmake ../fire -DCMAKE\_BUILD\_TYPE=Debug -G"Eclipse CDT4 - Unix Makefiles" -DCMAKE\_ECLIPSE\_VERSION=4.5
make
\end{DoxyCode}


If you would like to use M\+A\+G\+MA for solvers, you need to modify the cmake argument with the path to the M\+A\+G\+MA installation. Your configuration statement should look like the following\+:


\begin{DoxyCode}
cmake ../fire -DCMAKE\_BUILD\_TYPE=Debug -G"Eclipse CDT4 - Unix Makefiles" -DCMAKE\_ECLIPSE\_VERSION=4.5
       -DMAGMA\_ROOT=/usr/local/lib
\end{DoxyCode}
 Fire also supports C\+V\+O\+DE, which can be used by either passing -\/\+D\+S\+U\+N\+D\+I\+A\+L\+S\+\_\+\+R\+O\+OT or pointing to Spack\+:


\begin{DoxyCode}
cmake ../fire -DCMAKE\_BUILD\_TYPE=Debug -G"Eclipse CDT4 - Unix Makefiles" -DCMAKE\_ECLIPSE\_VERSION=4.5
       -DMAGMA\_ROOT=/usr/local/lib -DSPACK\_ROOT=$HOME/spack
\end{DoxyCode}


The above will get the code running, but it will not run the tests or generate the documentation. Issue the following commands to do that\+: 
\begin{DoxyCode}
make test
make doc
\end{DoxyCode}


Build flags, such as -\/\+Wall, can be set by prepending the C\+X\+X\+\_\+\+F\+L\+A\+GS variable to the cmake command as such


\begin{DoxyCode}
CXX\_FLAGS='-Wall' cmake ../fire -DCMAKE\_BUILD\_TYPE=Debug -G"Eclipse CDT4 - Unix Makefiles"
       -DCMAKE\_ECLIPSE\_VERSION=4.5
\end{DoxyCode}


Optimization flags should be handled by setting -\/\+D\+C\+M\+A\+K\+E\+\_\+\+B\+U\+I\+L\+D\+\_\+\+T\+Y\+PE=Release instead of Debug. Likewise, an optimized build with debug information can be acheived by setting -\/\+D\+C\+M\+A\+K\+E\+\_\+\+B\+U\+I\+L\+D\+\_\+\+T\+Y\+PE=Rel\+With\+Debug\+Info.

\subsection*{License}

See the L\+I\+C\+E\+N\+SE file licensing and copyright information. In short, 3-\/clause B\+SD.

\subsection*{Questions}

Questions can be directed to me at jayjaybillings $<$at$>$ gmail $<$dot$>$ com. 